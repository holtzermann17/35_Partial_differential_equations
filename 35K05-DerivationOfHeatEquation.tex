\documentclass[12pt]{article}
\usepackage{pmmeta}
\pmcanonicalname{DerivationOfHeatEquation}
\pmcreated{2013-03-22 18:45:04}
\pmmodified{2013-03-22 18:45:04}
\pmowner{pahio}{2872}
\pmmodifier{pahio}{2872}
\pmtitle{derivation of heat equation}
\pmrecord{9}{41528}
\pmprivacy{1}
\pmauthor{pahio}{2872}
\pmtype{Derivation}
\pmcomment{trigger rebuild}
\pmclassification{msc}{35K05}
\pmclassification{msc}{35Q99}
%\pmkeywords{Gauss theorem}
%\pmkeywords{Ostrogradsky theorem}
\pmrelated{DerivationOfWaveEquation}

\endmetadata

% this is the default PlanetMath preamble.  as your knowledge
% of TeX increases, you will probably want to edit this, but
% it should be fine as is for beginners.

% almost certainly you want these
\usepackage{amssymb}
\usepackage{amsmath}
\usepackage{amsfonts}

% used for TeXing text within eps files
%\usepackage{psfrag}
% need this for including graphics (\includegraphics)
%\usepackage{graphicx}
% for neatly defining theorems and propositions
 \usepackage{amsthm}
% making logically defined graphics
%%%\usepackage{xypic}

% there are many more packages, add them here as you need them

% define commands here

\theoremstyle{definition}
\newtheorem*{thmplain}{Theorem}

\begin{document}
Let us consider the heat conduction in a \PMlinkescapetext{homogeneous matter with density} $\varrho$ and specific heat capacity $c$.\, Denote by \,$u(x,\,y,\,z,\,t)$\, the temperature in the point \,$(x,\,y,\,z)$\, at the time $t$.\, Let $a$ be a \PMlinkescapetext{simple} \PMlinkescapetext{closed} surface in the matter and $v$ the spatial region \PMlinkescapetext{restricted} by it.

When the growth of the temperature of a volume element $dv$ in the time $dt$ is $du$, the element releases the amount 
$$-du\;c\,\varrho\,dv \;=\; -u'_t\,dt\,c\,\varrho\,dv$$
of heat, which is the heat flux through the surface of $dv$.\, Thus if there are no sources and sinks of heat in $v$, the heat flux through the surface $a$ in $dt$ is
\begin{align}
-dt\int_vc\varrho u'_t\,dv.
\end{align}
On the other hand, the flux through $da$ in the time $dt$ must be proportional to $a$, to $dt$ and to the derivative of the temperature in the direction of the normal line of the surface element $da$, i.e. the flux is
$$-k\,\nabla{u}\cdot d\vec{a}\;dt,$$
where $k$ is a positive \PMlinkescapetext{constant} (because the heat \PMlinkescapetext{flows} always from higher temperature to lower one).\, Consequently, the heat flux through the whole surface $a$ is
$$-dt\oint_ak\nabla{u}\cdot d\vec{a},$$
which is, by the Gauss's theorem, same as
\begin{align}
-dt\int_vk\,\nabla\cdot\nabla{u}\,dv \;=\; -dt\int_vk\,\nabla^2u\,dv.
\end{align}
Equating the expressions (1) and (2) and dividing by $dt$, one obtains
$$\int_vk\,\nabla^2u\,dv \;=\; \int_vc\,\varrho u'_t\,dv.$$
Since this equation is valid for any region $v$ in the matter, we infer that 
$$k\,\nabla^2u \;=\; c\,\varrho u'_t.$$
Denoting\, $\displaystyle\frac{k}{c\varrho} = \alpha^2$,\, we can write this equation as
\begin{align}
\alpha^2\nabla^2u \;=\; \frac{\partial u}{\partial t}.
\end{align}
This is the differential equation of heat conduction, first derived by Fourier.

\begin{thebibliography}{9}
\bibitem{K.V.}{\sc K. V\"ais\"al\"a:} {\em Matematiikka IV}.\, Handout Nr. 141.\quad Teknillisen korkeakoulun ylioppilaskunta, Otaniemi, Finland (1967).
\end{thebibliography}
%%%%%
%%%%%
\end{document}
