\documentclass[12pt]{article}
\usepackage{pmmeta}
\pmcanonicalname{WaveEquation}
\pmcreated{2013-03-22 13:10:12}
\pmmodified{2013-03-22 13:10:12}
\pmowner{Mathprof}{13753}
\pmmodifier{Mathprof}{13753}
\pmtitle{wave equation}
\pmrecord{10}{33614}
\pmprivacy{1}
\pmauthor{Mathprof}{13753}
\pmtype{Definition}
\pmcomment{trigger rebuild}
\pmclassification{msc}{35L05}
%\pmkeywords{partial differential equation}
\pmrelated{HelmholtzDifferentialEquation}
\pmrelated{SphericalMean}
\pmdefines{d'Alembert's solution to the wave equation}

\endmetadata

% this is the default PlanetMath preamble.  as your knowledge
% of TeX increases, you will probably want to edit this, but
% it should be fine as is for beginners.

% almost certainly you want these
\usepackage{amssymb}
\usepackage{amsmath}
\usepackage{amsfonts}

% used for TeXing text within eps files
%\usepackage{psfrag}
% need this for including graphics (\includegraphics)
%\usepackage{graphicx}
% for neatly defining theorems and propositions
%\usepackage{amsthm}
% making logically defined graphics
%%%\usepackage{xypic}

% there are many more packages, add them here as you need them

% define commands here
\begin{document}
The \emph{wave equation} is a partial differential equation which
describes certain kinds of waves.  It arises in various physical
situations, such as vibrating \PMlinkescapetext{strings}, \PMlinkescapetext{sound} waves, and
electromagnetic waves.

The wave equation in one \PMlinkescapetext{dimension} is
$$
\frac{\partial^2 u}{\partial t^2}=c^2\frac{\partial^2 u}{\partial
  x^2}.
$$
The general solution of the one-dimensional wave equation can be
obtained by a change of coordinates: $(x,t)\longrightarrow(\xi,\eta)$,
where $\xi=x-ct$ and $\eta=x+ct$.  This gives $\frac{\partial^2 u}{\partial\xi\partial\eta}=0$, which we can integrate to get \emph{d'Alembert's solution}:
$$
u(x,t)=F(x-ct)+G(x+ct)
$$
where $F$ and $G$ are twice differentiable functions.  $F$ and $G$
represent waves traveling in the positive and negative $x$
directions, respectively, with velocity $c$.  These functions can be
obtained if appropriate initial conditions and boundary conditions are given.  For example, if $u(x,0)=f(x)$ and $\frac{\partial u}{\partial t}(x,0)=g(x)$ are given, the solution is
$$
u(x,t)=\frac{1}{2}[f(x-ct)+f(x+ct)]+\frac{1}{2c}\int_{x-ct}^{x+ct}g(s)\mathrm d s.
$$

In general, the wave equation in $n$ \PMlinkescapetext{dimensions} is
$$
\frac{\partial^2 u}{\partial t^2}=c^2\nabla^2 u.
$$
where $u$ is a function of the location variables
$x_1,x_2,\ldots,x_n$, and time $t$.  Here, $\nabla^2$ is the Laplacian
with respect to the location variables, which in Cartesian coordinates is given by $
\nabla^2=\frac{\partial^2}{\partial x_1^2}+\frac{\partial^2}{\partial x_2^2}+\cdots+\frac{\partial^2}{\partial x_n^2}$.
%%%%%
%%%%%
\end{document}
