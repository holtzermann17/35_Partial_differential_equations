\documentclass[12pt]{article}
\usepackage{pmmeta}
\pmcanonicalname{DifferentialEquation}
\pmcreated{2013-03-22 12:41:22}
\pmmodified{2013-03-22 12:41:22}
\pmowner{rspuzio}{6075}
\pmmodifier{rspuzio}{6075}
\pmtitle{differential equation}
\pmrecord{11}{32969}
\pmprivacy{1}
\pmauthor{rspuzio}{6075}
\pmtype{Topic}
\pmcomment{trigger rebuild}
\pmclassification{msc}{35-00}
\pmclassification{msc}{34-00}
\pmrelated{HeatEquation}
\pmrelated{MethodOfUndeterminedCoefficients}
\pmrelated{ExampleOfUniversalStructure}
\pmrelated{CauchyInitialValueProblem}
\pmrelated{Equation}
\pmrelated{MaxwellsEquations}
\pmdefines{ordinary differential equation}
\pmdefines{ODE}
\pmdefines{partial differential equation}
\pmdefines{PDE}
\pmdefines{homogeneous}
\pmdefines{nonhomogeneous}
\pmdefines{inhomogeneous}
\pmdefines{linear differential equation}
\pmdefines{nonlinear differential equation}

% this is the default PlanetMath preamble.  as your knowledge
% of TeX increases, you will probably want to edit this, but
% it should be fine as is for beginners.

% almost certainly you want these
\usepackage{amssymb}
\usepackage{amsmath}
\usepackage{amsfonts}

% used for TeXing text within eps files
%\usepackage{psfrag}
% need this for including graphics (\includegraphics)
%\usepackage{graphicx}
% for neatly defining theorems and propositions
%\usepackage{amsthm}
% making logically defined graphics
%%%\usepackage{xypic}

% there are many more packages, add them here as you need them

% define commands here
\def\del{\partial}
\begin{document}
A {\em differential equation} is an equation involving an unknown function
of one or more variables, its derivatives and the
\PMlinkescapetext{independent} variables.
This type of equations comes up often in many different branches of
mathematics. They are also especially important in many problems in
physics and engineering.

There are many types of differential equations. An {\em ordinary differential
equation} (ODE) is a differential equation where the unknown function depends
on a single variable. A general ODE has the form
\begin{equation}
 \label{general-ode}
 F(x, f(x),f'(x),\ldots,f^{(n)}(x))=0,
\end{equation}
where the unknown $f$ is usually understood to be a real or complex valued function of $x$, and $x$ is usually understood to be either a real or complex
variable.
The {\em \PMlinkescapetext{order}}
of a differential equation is the order of the highest derivative appearing
in Eq. \eqref{general-ode}. In this case, assuming that $F$ depends
nontrivially on $f^{(n)}(x)$, the equation is of $n$th order.

If a differential equation is satisfied by a function which
identically vanishes (i.e. $f(x)=0$ for each $x$ in the
\PMlinkescapetext{domain} of interest),
then the equation is said to be {\em homogeneous}. Otherwise it is said to be
{\em nonhomogeneous} (or {\em inhomogeneous}). Many differential equations
can be expressed in the form \[ L[f] = g(x), \] where $L$ is a differential
operator (with $g(x)=0$ for the homogeneous case). If the operator $L$ is
linear in $f$, then the equation is said to be a {\em linear} ODE and
otherwise {\em nonlinear}.

Other types of differential equations involve more complicated \PMlinkescapetext{relations} involving the unknown function.
A {\em partial differential equation} (PDE) is a differential equation
where the unknown function depends on more than one variable. In a {\em delay differential equation} (DDE), the unknown function depends on the state of the system at some instant in the past.

Solving differential equations is a difficult task. Three major types of approaches are possible:
\begin{itemize}
\item {\em Exact} methods are generally \PMlinkescapetext{restricted} to equations of low order  and/or to linear systems. 
\item {\em Qualitative} methods do not give explicit \PMlinkescapetext{formula} for the solutions, but provide \PMlinkescapetext{information} pertaining to the asymptotic behavior of the system.
\item Finally, {\em numerical} methods allow to construct approximated solutions.
\end{itemize}

\subsection*{Examples}
A common example of an ODE is the equation for simple harmonic motion
\[ \frac{d^2u}{dx^2} + ku = 0. \]
This equation is of second order. It can be transformed into a system of two first order differential equations by introducing a variable $v=du/dx$. Indeed, we then have
\[
\aligned
\frac{dv}{dx} &= -ku \\
\frac{du}{dx} &= v.
\endaligned
\]

A common example of a PDE is the wave equation in three dimensions
\[ \frac{\del^2u}{\del x^2} + \frac{\del^2u}{\del y^2} +
 \frac{\del^2u}{\del z^2} = c^2 \frac{\del^2u}{\del t^2} \]
%%%%%
%%%%%
\end{document}
