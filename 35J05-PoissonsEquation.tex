\documentclass[12pt]{article}
\usepackage{pmmeta}
\pmcanonicalname{PoissonsEquation}
\pmcreated{2013-03-22 13:38:28}
\pmmodified{2013-03-22 13:38:28}
\pmowner{pbruin}{1001}
\pmmodifier{pbruin}{1001}
\pmtitle{Poisson's equation}
\pmrecord{6}{34291}
\pmprivacy{1}
\pmauthor{pbruin}{1001}
\pmtype{Definition}
\pmcomment{trigger rebuild}
\pmclassification{msc}{35J05}
%\pmkeywords{partial differential equation}
\pmrelated{HelmholtzDifferentialEquation}
\pmrelated{LaplacesEquation}
\pmrelated{GreensFunction}

\endmetadata

% this is the default PlanetMath preamble.  as your knowledge
% of TeX increases, you will probably want to edit this, but
% it should be fine as is for beginners.

% almost certainly you want these
\usepackage{amssymb}
\usepackage{amsmath}
\usepackage{amsfonts}

% used for TeXing text within eps files
%\usepackage{psfrag}
% need this for including graphics (\includegraphics)
%\usepackage{graphicx}
% for neatly defining theorems and propositions
%\usepackage{amsthm}
% making logically defined graphics
%%%\usepackage{xypic}

% there are many more packages, add them here as you need them

% define commands here
\begin{document}
\PMlinkescapeword{source}
\PMlinkescapeword{distribution}

Poisson's equation is a second-order partial differential equation which arises in physical problems such as finding the electric potential of a given charge distribution.  Its general form in $n$ \PMlinkescapetext{dimensions} is
$$
\nabla^2\phi(\mathbf r)=\rho(\mathbf r)
$$
where $\nabla^2$ is the Laplacian and $\rho:D\to\mathbb{R}$, often called a \emph{source \PMlinkescapetext{function}}, is a given function on some subset $D$ of $\mathbb{R}^n$.  If $\rho$ is identically zero, the Poisson equation reduces to the Laplace equation.

The Poisson equation is linear, and therefore obeys the \emph{superposition principle}: if $\nabla^2\phi_1=\rho_1$ and $\nabla^2\phi_2=\rho_2$, then $\nabla^2(\phi_1+\phi_2)=\rho_1+\rho_2$.  This fact can be used to construct solutions to Poisson's equation from \emph{fundamental solutions}, or \emph{Green's functions}, where the source distribution is a delta function.

A very important case is the one in which $n=3$, $D$ is all of $\mathbb{R}^3$, and $\phi(\mathbf r)\to 0$ as $\vert\mathbf r\vert\to\infty$.  The general solution is then given by
$$
\phi(\mathbf r)=-\frac{1}{4\pi}\int_{\mathbb{R}^3}\frac{\rho(\mathbf{r'})}{\vert\mathbf{r}-\mathbf{r'}\vert}\mathrm{d}^3\mathbf{r'}.
$$
%%%%%
%%%%%
\end{document}
