\documentclass[12pt]{article}
\usepackage{pmmeta}
\pmcanonicalname{DerivationOfWaveEquationFromMaxwellsEquations}
\pmcreated{2013-03-22 17:52:09}
\pmmodified{2013-03-22 17:52:09}
\pmowner{invisiblerhino}{19637}
\pmmodifier{invisiblerhino}{19637}
\pmtitle{derivation of wave equation from Maxwell's equations}
\pmrecord{10}{40348}
\pmprivacy{1}
\pmauthor{invisiblerhino}{19637}
\pmtype{Derivation}
\pmcomment{trigger rebuild}
\pmclassification{msc}{35Q60}
\pmclassification{msc}{78A25}
%\pmkeywords{wave equation}

% this is the default PlanetMath preamble.  as your knowledge
% of TeX increases, you will probably want to edit this, but
% it should be fine as is for beginners.

% almost certainly you want these
\usepackage{amssymb}
\usepackage{amsmath}
\usepackage{amsfonts}

% used for TeXing text within eps files
%\usepackage{psfrag}
% need this for including graphics (\includegraphics)
%\usepackage{graphicx}
% for neatly defining theorems and propositions
%\usepackage{amsthm}
% making logically defined graphics
%%%\usepackage{xypic}

% there are many more packages, add them here as you need them

% define commands here

\begin{document}
Maxwell was the first to note that Amp\`ere's Law does not satisfy conservation of charge (his corrected form is given in Maxwell's equation). This can be shown using the equation of conservation of electric charge:
\[
\nabla \cdot \mathbf{J} + \frac{\partial \rho}{\partial t} = 0
\]

Now consider Faraday's Law in differential form:
\[
\nabla \times \mathbf{E} = -\frac{ \partial \mathbf{B}}{\partial t}
\]
Taking the curl of both sides:
\[
\nabla \times (\nabla \times \mathbf{E}) =  \nabla \times (- \frac{ \partial \mathbf{B}}{\partial t})
\]

The right-hand side may be simplified by noting that
\[
\nabla \times (\frac{ \partial \mathbf{B}}{\partial t}) = - \frac{ \partial}{\partial t} (\nabla \times \mathbf{B})
\]
Recalling Amp\`ere's Law,
\[
- \frac{ \partial}{\partial t} (\nabla \times \mathbf{B}) = -\mu_0 \epsilon_0 \frac{ \partial^2 \mathbf{E}}{\partial t^2} 
\]
Therefore
\[
\nabla \times (\nabla \times \mathbf{E}) = -\mu_0 \epsilon_0 \frac{ \partial^2 \mathbf{E}}{\partial t^2} 
\]
The left hand side may be simplified by the following vector identity:
\[
\nabla \times (\nabla \times \mathbf{E}) = -\nabla^2 \mathbf{E}
\]
Hence
\[
\nabla^2 \mathbf{E} = \mu_0 \epsilon_0 \frac{ \partial^2 \mathbf{E}}{\partial t^2} 
\]
Applying the same analysis to Amp\'ere's Law then substituting in Faraday's Law leads to the result 
\[
\nabla^2 \mathbf{B} = \mu_0 \epsilon_0 \frac{ \partial^2 \mathbf{E}}{\partial t^2} 
\]
Making the substitution $\mu_0 \epsilon_0 = 1/c^2$ we note that these equations take the form of a transverse wave travelling at constant speed $c$. Maxwell evaluated the constants $\mu_0$ and $\epsilon_0$ according to their known values at the time and concluded that $c$ was approximately equal to 310,740,000 $\mbox{ms}^{-1}$, a value within ~3\% of today's results!
%%%%%
%%%%%
\end{document}
