\documentclass[12pt]{article}
\usepackage{pmmeta}
\pmcanonicalname{BeltramiDifferentialEquation}
\pmcreated{2013-03-22 14:08:34}
\pmmodified{2013-03-22 14:08:34}
\pmowner{jirka}{4157}
\pmmodifier{jirka}{4157}
\pmtitle{Beltrami differential equation}
\pmrecord{8}{35557}
\pmprivacy{1}
\pmauthor{jirka}{4157}
\pmtype{Definition}
\pmcomment{trigger rebuild}
\pmclassification{msc}{35F20}
\pmclassification{msc}{30C62}
\pmrelated{QuasiconformalMapping}

\endmetadata

% this is the default PlanetMath preamble.  as your knowledge
% of TeX increases, you will probably want to edit this, but
% it should be fine as is for beginners.

% almost certainly you want these
\usepackage{amssymb}
\usepackage{amsmath}
\usepackage{amsfonts}

% used for TeXing text within eps files
%\usepackage{psfrag}
% need this for including graphics (\includegraphics)
%\usepackage{graphicx}
% for neatly defining theorems and propositions
%\usepackage{amsthm}
% making logically defined graphics
%%%\usepackage{xypic}

% there are many more packages, add them here as you need them

% define commands here
\begin{document}
Suppose that $\mu : G \subset {\mathbb{C}} \rightarrow {\mathbb{C}}$ is a measurable function, then the partial differential equation
\begin{equation*}
f_{\bar{z}}(z) = \mu(z)f_z(z)
\end{equation*}
is called the {\em Beltrami differential equation}.

If furthermore $\lvert \mu(z) \rvert < 1$ and
in fact $\lvert \mu(z) \rvert$ has a uniform bound less then 1 over the domain of definition, then the solution is a quasiconformal mapping with \PMlinkname{complex dilation}{QuasiconformalMapping} $\mu(z)$ and \PMlinkname{maximal small dilatation}{QuasiconformalMapping} $d_f = \sup_z \lvert \mu(z) \rvert$.

A conformal mapping has $f_{\bar{z}} \equiv 0$ and so the solution can be conformal if and only if $\mu \equiv 0$.

The partial derivatives $f_z$ and $f_{\bar{z}}$ (where $\bar{z}$ is the complex conjugate of $z$) can here be given in terms of
the real and imaginary parts of $f = u + iv$ as
\begin{align*}
f_z & = \frac{1}{2} ( u_x + v_y ) + \frac{i}{2} ( v_x - u_y ),
\\
f_{\bar{z}} & = \frac{1}{2} ( u_x - v_y ) + \frac{i}{2} ( v_x + u_y ).
\end{align*}

\begin{thebibliography}{9}
\bibitem{ahlfors}
L.\@ V.\@ Ahlfors.  \emph{\PMlinkescapetext{Lectures on Quasiconformal
Mappings}}.  Van Nostrand-Reinhold, Princeton, New Jersey, 1966
\end{thebibliography}
%%%%%
%%%%%
\end{document}
