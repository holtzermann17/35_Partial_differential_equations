\documentclass[12pt]{article}
\usepackage{pmmeta}
\pmcanonicalname{MethodOfIntegratingFactors}
\pmcreated{2013-03-22 16:31:48}
\pmmodified{2013-03-22 16:31:48}
\pmowner{pahio}{2872}
\pmmodifier{pahio}{2872}
\pmtitle{method of integrating factors}
\pmrecord{21}{38710}
\pmprivacy{1}
\pmauthor{pahio}{2872}
\pmtype{Topic}
\pmcomment{trigger rebuild}
\pmclassification{msc}{35-00}
\pmclassification{msc}{34-00}
\pmrelated{ErnstLindelof}
\pmdefines{integrating factor}
\pmdefines{Euler multiplicator}

% this is the default PlanetMath preamble.  as your knowledge
% of TeX increases, you will probably want to edit this, but
% it should be fine as is for beginners.

% almost certainly you want these
\usepackage{amssymb}
\usepackage{amsmath}
\usepackage{amsfonts}

% used for TeXing text within eps files
%\usepackage{psfrag}
% need this for including graphics (\includegraphics)
%\usepackage{graphicx}
% for neatly defining theorems and propositions
 \usepackage{amsthm}
% making logically defined graphics
%%%\usepackage{xypic}

% there are many more packages, add them here as you need them

% define commands here

\theoremstyle{definition}
\newtheorem*{thmplain}{Theorem}

\begin{document}
\PMlinkescapeword{connection} \PMlinkescapeword{equivalent}

The {\em method of integrating factors} is in principle a means for solving ordinary differential equations of first \PMlinkescapetext{order}.\, It has not great practical significance, but is theoretically important.\\

Let us consider a differential equation solved for the derivative $y'$ of the unknown function and write the equation in the form
\begin{align}
  X(x,\,y)\,dx+Y(x,\,y)\,dy \;=\; 0.
\end{align}
We assume that the functions $X$ and $Y$ have continuous partial derivatives in a region $R$ of $\mathbb{R}^2$.

If there is a solution of (1) which may be expressed in the form
  $$f(x,\,y) \;=\; C$$
with $f$ having continuous partial derivatives in $R$ and with $C$ an arbitrary constant, then it's not difficult to see that such an $f$ satisfies the linear partial differential equation
\begin{align}
  X\frac{\partial f}{\partial y}-Y\frac{\partial f}{\partial x} \;=\; 0.
\end{align}
Conversely, every non-constant solution $f$ of (2) gives also a solution\, $f(x,\,y) = C$\, of (1).\, Thus, solving (1) and solving (2) are \PMlinkname{equivalent}{Equivalent3} tasks.

It's straightforward to show that if\, $f_0(x,\,y)$\, is a non-constant solution of the equation (2), then all solutions of this equation are\, $F(f_0(x,\,y))$\, where $F$ is a freely chosen function with (mostly) continuous derivative.

The connection of the equations (1) and (2) may be presented also in another form.\, Suppose that\, $f(x,\,y) = C$\, is any solution of (1).\, Then (2) implies  the proportion equation
  $$\frac{f_x'}{X} \;=\; \frac{f_y'}{Y}.$$
If we denote the common value of these two ratios by\, $\mu(x,\,y) = \mu$,\, then we have
  $$f_x' \;=\; \mu X, \qquad  f_y' \;=\; \mu Y.$$
This gives to the differential of the function $f$ the expression
  $$d\,f(x,\,y) \;=\; \mu(x,\,y)(X(x,\,y)\,dx+Y(x,\,y)\,dy).$$
We see that\, $\mu(x,\,y)$\, is the {\em integrating factor} or {\em Euler multiplicator} of the given differential equation (1), i.e. the left hand side of (1) turns, when multiplied by\, $\mu(x,\,y)$,\, to an \PMlinkname{exact differential}{ExactDifferentialForm}.

Conversely, any integrating factor $\mu$ of (1), i.e. such that\, 
$\mu X\,dx+\mu Y\,dy$\, is the differential of some function $f$, is easily seen to determine the solutions of the form\, $f(x,\,y) = C$\, of (1).\, Altogether, solving the differential equation (1) is equivalent with finding an integrating factor of the equation.

When an integrating factor $\mu$ of (1) is available, the solution function $f$
can be gotten from the line integral
$$f(x,\,y) \;=:\; \int_{P_0}^P [\mu(x,\,y)X(x,\,y)\,dx+\mu(x,\,y)Y(x,\,y)\,dy]$$
along any curve $\gamma$ connecting an arbitrarily chosen point \,$P_0 =(x_0,\,y_0)$\, and the point\, $P = (x,\,y)$\, in the region $R$.\\

\textbf{Note.}\, In general, it's very hard to find a suitable integrating factor.\, One special case where such can be found, is that $X$ and $Y$ are homogeneous functions of same \PMlinkname{degree}{HomogeneousFunction}: then the expression $\displaystyle\frac{1}{xX+yY}$ is an integrating factor.\\

\textbf{Example.}\, In the differential equation
    $$(x^4+y^4)\,dx-xy^3\,dy \;=\; 0$$
we see that\, $X =: x^4+y^4$\, and\, $Y =: -xy^3$\, both define a \PMlinkname{homogeneous function of degree}{HomogeneousFunction} 4.\, Thus we have the integrating factor\, $\displaystyle\mu =: \frac{1}{x^5+xy^4-xy^4} = \frac{1}{x^5}$,\, and the left hand side of the equation
    $$\left(\frac{1}{x}+\frac{y^4}{x^5}\right)\,dx-\frac{y^3}{x^4}\,dy \;=\; 0$$
is an exact differential.\, We can integrate it along the broken line, first from\, $(1,\,0)$\, to\, $(x,\,0)$\, and then still to\, $(x,\,y)$,\, obtaining
$$f(x,\,y) \;=:\; \int_1^x\left(\frac{1}{x}+\frac{0^4}{x^5}\right)\,dx
-\int_0^y\frac{y^3\,dy}{x^4} \;=\; \ln|x|-\frac{y^4}{4x^4}.$$
So the general solution of the given differential equation is
$$\ln|x|-\frac{y^4}{4x^4} \;=\; C.$$


\begin{thebibliography}{9}
\bibitem{3L}{\sc E. Lindel\"of:} {\em Differentiali- ja integralilasku III 1}.\, Mercatorin Kirjapaino Osakeyhti\"o, Helsinki (1935).
\end{thebibliography}

%%%%%
%%%%%
\end{document}
