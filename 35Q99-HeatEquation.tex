\documentclass[12pt]{article}
\usepackage{pmmeta}
\pmcanonicalname{HeatEquation}
\pmcreated{2013-03-22 12:45:36}
\pmmodified{2013-03-22 12:45:36}
\pmowner{drini}{3}
\pmmodifier{drini}{3}
\pmtitle{heat equation}
\pmrecord{5}{33067}
\pmprivacy{1}
\pmauthor{drini}{3}
\pmtype{Definition}
\pmcomment{trigger rebuild}
\pmclassification{msc}{35Q99}
\pmrelated{DifferentialEquation}
\pmrelated{Laplacian}

% this is the default PlanetMath preamble.  as your knowledge
% of TeX increases, you will probably want to edit this, but
% it should be fine as is for beginners.

% almost certainly you want these
\usepackage{amssymb}
\usepackage{amsmath}
\usepackage{amsfonts}

% used for TeXing text within eps files
%\usepackage{psfrag}
% need this for including graphics (\includegraphics)
%\usepackage{graphicx}
% for neatly defining theorems and propositions
%\usepackage{amsthm}
% making logically defined graphics
%%%\usepackage{xypic} 

% there are many more packages, add them here as you need them

% define commands here
\begin{document}
The heat equation in 1-dimension (for example, along a metal wire) is a partial differential equation of the following form:

$$\frac{\partial u}{\partial t} = c^2 \cdot \frac{\partial^2 u}{\partial x^2}$$

also written as

$$u_{t} = c^2 \cdot u_{xx}$$

Where $u:\mathbb{R}^2\to\mathbb{R}$ is the function giving the temperature at time $t$ and position $x$ and $c$ is a real valued constant. This can be easily extended to 2 or 3 dimensions as

$$u_{t} = c^2 \cdot ( u_{xx} + u_{yy} )$$
and
$$u_{t} = c^2 \cdot ( u_{xx} + u_{yy} + u_{zz} )$$

Note that in the steady state, that is when $u_{t} = 0$, we are left with the Laplacian of $u$:

$$\Delta u = 0 $$
%%%%%
%%%%%
\end{document}
